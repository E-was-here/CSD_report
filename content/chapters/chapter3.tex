\chapter{LQR Controller}

\section{State Space Model}
In control system design, it is often easier to define a parameterized state-space model in continuous time because physical laws are typically described using differential equations. The linear state-space representation in continuous-time has the following form:

\begin{equation}
    \dot{\mathbf{x}} = \mathbf{A} \mathbf{x} + \mathbf{B} \mathbf{u}
    \label{eq:state_space_models}
\end{equation}

where \( \mathbf{x} \) is the state vector, \( \mathbf{A} \) is the state matrix that represents the system dynamics, \( \mathbf{B} \) is the input matrix that represents the control input, and \( \mathbf{u} \) is the input vector.
The state-space representation can be explicitly defined as:

\begin{equation}
    \dot{\mathbf{x}}(t) = \begin{bmatrix}
    \frac{-1}{\tau_1} & 0 \\ 
    0 & \frac{-1}{\tau_2}
    \end{bmatrix} \mathbf{x}(t) + \begin{bmatrix} 
    \frac{A_{01}}{\tau_1} \\ 
    \frac{A_{02}}{\tau_2} 
    \end{bmatrix}\mathbf{u}(t)
    \label{eq:state_space_model_cx}
\end{equation}

The output equation is given by:

\begin{equation}
    \mathbf{y}(t) = \begin{bmatrix} 1 & 1 \end{bmatrix}\mathbf{x}(t) + \begin{bmatrix} 0 & 0 \end{bmatrix}\mathbf{u}(t)
    \label{eq:state_space_model_cy}
\end{equation}

where \( \mathbf{y} \) is the output vector, which also comprises the shaft angular velocity \( \omega \) and the motor voltage \( V \).

\section{Discrete-Time State-Space Model}
We convert the continuous-time state-space model \eqref{eq:state_space_model_cx} and \eqref{eq:state_space_model_cy} to a discrete-time state-space model using the parameters:

\[
\tau_1 = 1.915, \quad \tau_2 = 1.7, \quad A_{01} = 24.88, \quad A_{02} = 19.51.
\]

The continuous-time matrices are computed as:

\[
\mathbf{A} = 
\begin{bmatrix}
-0.5218 & 0 \\
0 & -0.5882
\end{bmatrix}, \quad 
\mathbf{B} = 
\begin{bmatrix}
12.9870 \\
11.4765
\end{bmatrix}.
\]

For a sampling time \( T_s = 0.1 \, \text{seconds} \), the discrete-time matrices \( A_{T_s} \) and \( B_{T_s} \) are calculated as follows:

\begin{itemize}
    \item \textbf{Discrete-Time State Matrix} (\(A_{T_s}\)): 
   Using the matrix exponential formula:
   \[
   A_{T_s} = e^{A T_s},
   \]
   where:
   \[
   A T_s = 
   \begin{bmatrix}
   -0.05218 & 0 \\
   0 & -0.05882
   \end{bmatrix}.
   \]
   Since \( A T_s \) is a diagonal matrix, the matrix exponential is computed as:
   \[
   e^{A T_s} = 
   \begin{bmatrix}
   e^{-0.05218} & 0 \\
   0 & e^{-0.05882}
   \end{bmatrix}.
   \]
   Evaluating the exponentials gives:
   \[
   e^{-0.05218} \approx 0.9491, \quad e^{-0.05882} \approx 0.9429.
   \]
   Therefore, we find:
   \[
   A_{T_s} = 
   \begin{bmatrix}
   0.9491 & 0 \\
   0 & 0.9429
   \end{bmatrix}.
   \]
\end{itemize}

\begin{itemize} 
    \item \textbf{Discrete-Time Input Matrix} (\(B_{T_s}\)): 
   Using the formula:
   \[
   B_{T_s} = \int_0^{T_s} e^{A \zeta} B \, d\zeta,
   \]
   and solving numerically, we find:
   \[
   B_{T_s} = 
   \begin{bmatrix}
   1.2662 \\
   1.1149
   \end{bmatrix}.
   \]
\end{itemize}

Thus, the discrete-time state-space representation of the system is:

\[
x[k+1] = A_{T_s} x[k] + B_{T_s} u[k],
\]
\[
y[k] = C x[k],
\]

where:

\[
A_{T_s} = 
\begin{bmatrix}
0.9491 & 0 \\
0 & 0.9429
\end{bmatrix}, \quad
B_{T_s} = 
\begin{bmatrix}
1.2662 \\
1.1149
\end{bmatrix}, \quad
C = 
\begin{bmatrix}
1 & 1
\end{bmatrix}.
\]

Finally, the sampled system is equivalent to a discrete-time system with these matrices. We will now use this representation to design an LQR controller for the given discrete-time system.

\section{LQR Design for Discrete-Time System}

The goal of the Linear Quadratic Regulator (LQR) is to minimize the quadratic cost function:

\[
J = \sum_{k=0}^\infty \left( x[k]^\top Q x[k] + u[k]^\top R u[k] \right).
\]

The weighting matrices \( Q \) and \( R \) must be selected based on system requirements:

\begin{itemize}
    \item \( Q \): Penalizes state deviations, typically represented as a diagonal matrix, e.g., \( Q = \text{diag}(q_1, q_2) \).
    \item \( R \): Penalizes large control efforts, usually a positive scalar, \( R > 0 \).
\end{itemize}

For our design, we assume the following weighting matrices:

\[
Q = \begin{bmatrix}
10 & 0 \\
0 & 1
\end{bmatrix}, \quad R = 1.
\]

The Discrete-Time Algebraic Riccati Equation (DARE) is given by:

\[
P = A_{T_s}^\top P A_{T_s} - \left(A_{T_s}^\top P B_{T_s} \right) 
\left(R + B_{T_s}^\top P B_{T_s} \right)^{-1} 
\left(B_{T_s}^\top P A_{T_s} \right) + Q.
\]

The optimal feedback gain matrix \( K \) is computed as:

\[
K = \left( R + B_{T_s}^\top P B_{T_s} \right)^{-1} \left(B_{T_s}^\top P A_{T_s} \right).
\]

The LQR controller generates the optimal control input:

\[
u[k] = -K x[k],
\]

where \( K \) is the gain matrix. We will compute \( K \) using MATLAB.

\subsection*{LQR Feedback Law}
The two motors share control effort based on these gains. If the shaft angular velocities are \( w_{1} \) and \( w_{2} \), then the control law is:

\[
u[k] = -K x[k], \quad u_{1} = -K_{1} w_{1}, \quad u_{2} = -K_{2} w_{2}.
\]

The computation results yield:

\[
K = \begin{bmatrix}
0.6618 \\ 0.0548
\end{bmatrix}, \quad u_{1} = 0.6618 w_{1}, \quad u_{2} = 0.0548 w_{2}.
\]
