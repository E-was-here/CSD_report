\setcounter{secnumdepth}{2}

\chapter{Design of the controller}

\section{Requirements analysis}

The first step of the controller design is to analyse the requirements that have been given in the introduction:

\begin{enumerate}
    \item[$\bullet$] the shaft can reach any reasonnable speed in less than half a second
    \item[$\bullet$] the shaft's speed in maintained as smooth as possible
    \item[$\bullet$] the system can reject a step disturbance as fast as possible
    \item[$\bullet$] the shaft's speed can follow a feasible speed reference of $4 Hz$ from any feasible speed
\end{enumerate}

There is one reference tracking objective, one disturbance rejection objective and two dynamical response properties.
The design of the controller will follow the following path:

\begin{enumerate}
    \item Rejecting a step disturbance (with each motor separately)
    \item Following a $4 Hz$ reference (with each motor separately)
    \item Merge the two controllers into a single digital one
    \item Reach any speed in $< 0.5 s$
\end{enumerate}

The method used here is to design the controller in there Laplace domain in the first time. Once the continuous time
controller has the desired behaviour, it is discretized using the Tustin method (and without forgetting to take the
ZOH into account).

\section{Disturbance rejection}

From theory, it is known that a controller is able to asymptotically reject a disturbance if it has at the denominator
of its transfer function the denominator of the disturbance.\\

As a step disturbance is of the form:

\begin{equation}
    D(s) = \frac{A e^{-\tau s}}{s}
\end{equation}

The controller will need a pole in $0$. This can be easily done by using a PI controller, which looks like:

\begin{align}
    C_{PI}(s) &= k_p \left( 1 + \frac{1}{T_i s} \right)\\
    &= k_p \left(\frac{T_i s + 1}{T_i s} \right)
    \label{eq:general_PI}
\end{align}

The $k_p$ will be chosen once the dynamic response characteristics will have to be met. However, $T_i$ can already be 
chosen based on a simple criteria. As the open-loop transfer function equals the product $C_{PI}(s) \times G_{1+} (s)$, a
solid choice is to use $T_i$ to cancel the pole in the system. This way, the position of the poles of the closed-loop
will be entirely based on the controller.\\
Of course this will be true in the case of a perfect model but as proved in the section \ref{section_validation}, the
parameter $\tau$ of the transfer function (which fixes the pole of it) don't seems to move a lot depending on the 
operating point. We can conclude from this that the best choice is:

\begin{equation}
    T_i = \tau
\end{equation}

for each controller, which leads to:

\begin{equation}
    C_{PI}(s) = k_p \left( \frac{\tau s + 1}{\tau s} \right) 
    \label{eq:controller_PI}
\end{equation}

\section{Reference tracking}

For tracking purposes, the method used to reject disturbances can also be applied. Indeed, a reference with $D(s)$ at
its denominator will be perfectly followed if the controller denominator contains $D(s)$.

\begin{align}
    R(s) &= \mathcal{L}\left\{A \sin (8 \pi t)\right\} \\
    &= \frac{8 A \pi}{s^2 + 64\pi^2}
\end{align}

A second controller is needed to ensure the tracking of a $4 Hz$ sine:

\begin{equation}
    C_{\sin}(s) = \frac{1}{s^2 + 64\pi^2}
    \label{eq:controller_sin}
\end{equation}

This controller has no parameter that can be moved as we chose a minimalist controller. 

\section{Merging and discretization of the controllers}

The total continuous controller is the product of the two previously built controllers because they are put in series, 
which gives a the following structure:

\begin{figure}[H]
    \centering
    \includegraphics[width=\textwidth]{Pictures/controller_structure.png}
    \caption{Structure of the combined \textit{continuous time} controller}
    \label{fig:controller_structure}
\end{figure}

